\documentclass[12pt,a4paper]{article}
\usepackage{fontspec}
\usepackage{xeCJK}
\usepackage{geometry}
\usepackage{enumitem}
\usepackage{listings}
\usepackage{xcolor}
\usepackage{booktabs}
\usepackage{array}
\usepackage{longtable}

% 页面设置
\geometry{margin=2.5cm}

% 代码块设置
\lstset{
    basicstyle=\ttfamily\small,
    backgroundcolor=\color{gray!10},
    frame=single,
    breaklines=true,
    showstringspaces=false,
    commentstyle=\color{green!60!black},
    keywordstyle=\color{blue},
    stringstyle=\color{red}
}

% 中文字体设置
\setCJKmainfont{PingFang SC}
\setCJKsansfont{PingFang SC}
\setCJKmonofont{PingFang SC}

\title{Kubernetes 编排工具问题解答}
\author{In-Class Assignment 2}
\date{\today}

\begin{document}

\maketitle

\section{问题1:编排工具的关键作用}

\textbf{问题:} Orchestration tools, such as Kubernetes, play a key role in the server infrastructure for the modern applications.

(a) Explain how these tools help manage and scale application servers.

(b) Describe how orchestration tools facilitate automated deployment, scaling, and management of application servers.

\textbf{答案:}

\subsection{(a) 编排工具如何帮助管理和扩展应用服务器}

\begin{itemize}
    \item \textbf{资源管理}:自动分配和管理计算资源(CPU、内存、存储),确保应用获得所需资源
    \item \textbf{负载均衡}:在多个服务器实例间分配流量,提高性能和可用性
    \item \textbf{健康监控}:持续监控应用状态,自动重启失败的实例
    \item \textbf{水平扩展}:根据负载自动增加或减少服务器实例数量
    \item \textbf{服务发现}:自动管理服务间的网络通信和依赖关系
\end{itemize}

\subsection{(b) 编排工具如何促进自动化部署、扩展和管理}

\begin{itemize}
    \item \textbf{声明式配置}:通过YAML文件定义期望状态,系统自动维护该状态
    \item \textbf{滚动更新}:无停机时间地更新应用版本
    \item \textbf{自动扩缩容}:基于CPU使用率、内存或自定义指标自动调整实例数量
    \item \textbf{故障恢复}:自动检测并替换失败的节点或容器
    \item \textbf{配置管理}:集中管理应用配置和密钥
\end{itemize}

\section{问题2:Pod、Deployment和Service的区别}

\textbf{问题:} Explain the difference between a Pod, Deployment, and Service.

\textbf{答案:}

\begin{itemize}
    \item \textbf{Pod}:
    \begin{itemize}
        \item Kubernetes中最小的可部署单元
        \item 包含一个或多个紧密耦合的容器
        \item 共享网络和存储资源
        \item 生命周期短暂,可能随时被删除和重建
    \end{itemize}
    
    \item \textbf{Deployment}:
    \begin{itemize}
        \item 管理Pod副本的高级控制器
        \item 确保指定数量的Pod副本始终运行
        \item 支持滚动更新和回滚
        \item 提供声明式的应用部署和管理
    \end{itemize}
    
    \item \textbf{Service}:
    \begin{itemize}
        \item 为Pod提供稳定的网络访问入口
        \item 通过标签选择器将流量路由到后端Pod
        \item 提供负载均衡功能
        \item 即使Pod重建,Service的IP和DNS名称保持不变
    \end{itemize}
\end{itemize}

\section{问题3:Kubernetes中的Namespace}

\textbf{问题:} What is a Namespace in Kubernetes? Please list one example.

\textbf{答案:}

\textbf{Namespace}是Kubernetes中的虚拟集群概念,用于在同一物理集群中创建多个逻辑隔离的环境。它提供:
\begin{itemize}
    \item 资源隔离和组织
    \item 访问控制和权限管理
    \item 资源配额限制
    \item 命名空间内的资源名称唯一性
\end{itemize}

\textbf{示例:}
\begin{lstlisting}
apiVersion: v1
kind: Namespace
metadata:
  name: development
\end{lstlisting}

常见的默认namespace包括:
\begin{itemize}
    \item \texttt{default}:默认命名空间
    \item \texttt{kube-system}:系统组件命名空间
    \item \texttt{kube-public}:公共资源命名空间
\end{itemize}

\section{问题4:Kubelet的作用和检查节点的命令}

\textbf{问题:} Explain the role of the Kubelet. How do you check the nodes in a Kubernetes cluster? (kubectl command expected)

\textbf{答案:}

\subsection{Kubelet的作用}
\begin{itemize}
    \item 运行在每个节点上的主要代理程序
    \item 负责管理节点上的Pod生命周期
    \item 与API Server通信,接收Pod规范
    \item 监控容器健康状态并报告给控制平面
    \item 管理容器的启动、停止和重启
    \item 执行健康检查和资源监控
\end{itemize}

\subsection{检查集群节点的kubectl命令}
\begin{lstlisting}[language=bash]
# 查看所有节点
kubectl get nodes

# 查看节点详细信息
kubectl get nodes -o wide

# 查看特定节点的详细描述
kubectl describe node <node-name>

# 查看节点状态
kubectl get nodes --show-labels
\end{lstlisting}

\section{问题5:ClusterIP、NodePort和LoadBalancer服务的区别}

\textbf{问题:} What is the difference between ClusterIP, NodePort, and LoadBalancer services?

\textbf{答案:}

\begin{itemize}
    \item \textbf{ClusterIP}:
    \begin{itemize}
        \item 默认服务类型
        \item 只能在集群内部访问
        \item 分配一个集群内部IP地址
        \item 适用于内部服务通信
    \end{itemize}
    
    \item \textbf{NodePort}:
    \begin{itemize}
        \item 在每个节点上开放一个特定端口(30000-32767)
        \item 可以通过任何节点的IP:NodePort访问服务
        \item 自动创建ClusterIP
        \item 适用于开发和测试环境
    \end{itemize}
    
    \item \textbf{LoadBalancer}:
    \begin{itemize}
        \item 创建外部负载均衡器(需要云提供商支持)
        \item 自动创建NodePort和ClusterIP
        \item 提供外部可访问的IP地址
        \item 适用于生产环境的外部访问
    \end{itemize}
\end{itemize}

\section{问题6:使用kubectl扩展Deployment到5个副本}

\textbf{问题:} How do you scale a Deployment to 5 replicas using kubectl?

\textbf{答案:}
\begin{lstlisting}[language=bash]
# 方法1:使用scale命令
kubectl scale deployment <deployment-name> --replicas=5

# 方法2:使用patch命令
kubectl patch deployment <deployment-name> -p '{"spec":{"replicas":5}}'

# 验证扩展结果
kubectl get deployment <deployment-name>
\end{lstlisting}

\section{问题7:无停机更新Deployment镜像}

\textbf{问题:} How would you update the image of a Deployment without downtime?

\textbf{答案:}
\begin{lstlisting}[language=bash]
# 方法1:使用set image命令
kubectl set image deployment/<deployment-name> <container-name>=<new-image>

# 方法2:使用patch命令
kubectl patch deployment <deployment-name> -p '{"spec":{"template":{"spec":{"containers":[{"name":"<container-name>","image":"<new-image>"}]}}}}'

# 查看滚动更新状态
kubectl rollout status deployment/<deployment-name>

# 如需回滚
kubectl rollout undo deployment/<deployment-name>
\end{lstlisting}

\section{问题8:将Deployment暴露给外部流量}

\textbf{问题:} How do you expose a Deployment to external traffic?

\textbf{答案:}
\begin{lstlisting}[language=bash]
# 方法1:创建NodePort服务
kubectl expose deployment <deployment-name> --type=NodePort --port=80

# 方法2:创建LoadBalancer服务
kubectl expose deployment <deployment-name> --type=LoadBalancer --port=80

# 方法3:使用kubectl create service
kubectl create service nodeport <service-name> --tcp=80:8080

# 方法4:通过Ingress(需要先创建Service)
kubectl expose deployment <deployment-name> --port=80
# 然后创建Ingress资源
\end{lstlisting}

\section{问题9:Kubernetes调度决策机制}

\textbf{问题:} How does Kubernetes scheduling decide which node a Pod runs on?

\textbf{答案:}

Kubernetes调度器通过以下步骤决定Pod运行的节点:

\subsection{调度过程}
\begin{enumerate}
    \item \textbf{过滤阶段}:排除不符合条件的节点
    \begin{itemize}
        \item 资源需求(CPU、内存)
        \item 节点选择器和亲和性规则
        \item 污点和容忍度
        \item 端口冲突检查
    \end{itemize}
    
    \item \textbf{评分阶段}:对符合条件的节点打分
    \begin{itemize}
        \item 资源利用率均衡
        \item 亲和性偏好
        \item 镜像本地性
        \item 节点负载均衡
    \end{itemize}
    
    \item \textbf{选择阶段}:选择得分最高的节点
\end{enumerate}

\subsection{影响因素}
\begin{itemize}
    \item 节点资源可用性
    \item Pod的资源请求和限制
    \item 节点选择器(nodeSelector)
    \item 亲和性和反亲和性规则
    \item 污点(Taints)和容忍度(Tolerations)
\end{itemize}

\section{问题10:Ingress的作用及其与Service的区别}

\textbf{问题:} What is the role of Ingress and how does it differ from a Service?

\textbf{答案:}

\subsection{Ingress的作用}
\begin{itemize}
    \item 管理集群外部对服务的HTTP/HTTPS访问
    \item 提供负载均衡、SSL终止和基于名称的虚拟主机
    \item 支持路径路由和域名路由
    \item 集中管理外部访问规则
\end{itemize}

\subsection{Ingress与Service的区别}

\begin{longtable}{|p{3cm}|p{5cm}|p{5cm}|}
\hline
\textbf{特性} & \textbf{Service} & \textbf{Ingress} \\
\hline
\textbf{层级} & L4(传输层) & L7(应用层) \\
\hline
\textbf{协议} & TCP/UDP & HTTP/HTTPS \\
\hline
\textbf{路由} & 基于端口 & 基于路径/域名 \\
\hline
\textbf{SSL} & 不支持 & 支持SSL终止 \\
\hline
\textbf{负载均衡} & 简单轮询 & 高级负载均衡 \\
\hline
\textbf{成本} & 每个服务需要LoadBalancer & 单个入口点 \\
\hline
\end{longtable}

\subsection{示例配置}
\begin{lstlisting}
apiVersion: networking.k8s.io/v1
kind: Ingress
metadata:
  name: example-ingress
spec:
  rules:
  - host: example.com
    http:
      paths:
      - path: /app1
        pathType: Prefix
        backend:
          service:
            name: app1-service
            port:
              number: 80
\end{lstlisting}

\end{document}